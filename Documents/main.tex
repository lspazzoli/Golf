\documentclass[english]{article}

\usepackage{graphicx}
\usepackage[T1]{fontenc}
\usepackage{babel}

\author{
	Maree, Armand\\
	\texttt{12017800}
	\and
	Watt, Brenton\\
	\texttt{XXXXXXXX}
	\and
	XXX, Charl\\
	\texttt{XXXXXXXX}
	\and
	XXX, Elton\\
	\texttt{XXXXXXXX}
	\and
	XXX, Kamogelo\\
	\texttt{XXXXXXXX}
	\and
	XXX, Keletso\\
	\texttt{XXXXXXXX}
	\and
	Spazolli, Lorenzo\\
	\texttt{XXXXXXXX}
}

\title{COS 301 Mini project}
\date{\today}

\begin{document}
	\maketitle
	\begin{figure}[!t]
		\includegraphics[width=\linewidth]{up_logo.png}
	\end{figure}
	\pagenumbering{gobble}
	\newpage
	\tableofcontents
	\newpage
	\pagenumbering{arabic}
	
	\section{Introduction}
		
		In this document it would be specified how a system would be developed for the department of computer science at the University of Pretoria in order for them to replace their current, not so efficient Microsoft Office Excel spreadsheet, system with a more concurrent and reliable option.

	\section{Vision}
	
		The client requires a system that would allow them to retrieve and submit meta-data about academic papers published by the department of computer science at the University of Pretoria. Certain users would be able to create new papers and the project leader would then be able to control certain properties of the project, like the current progress of the paper. It should also have a web interface where changes can be made and also an Android app.

	\section{Background}
	
	\section{Architecture Requirements}.
		\subsection{Access channel requirements}

		\subsection{Quality requirements}

		\subsection{Integration requirements}

		\subsection{Architecture constraints}

	\section{Functional requirements and application design}
		\subsection{Use case prioritization}

		\subsection{Use case/Services contracts}

		\subsection{Required functionality}

		\subsection{Process specifications}

		\subsection{Domain Model}

	\section{Open Issues}
\end{document}
